\documentclass[runningheads,a4paper]{llncs}

\usepackage[utf8]{inputenc}
\usepackage[OT1]{fontenc}
\usepackage[english]{babel}

\usepackage{amsmath}
\usepackage{amsfonts}
\usepackage{amssymb}
%\usepackage{amsthm}
\usepackage{amssymb}
\usepackage{bm}

\usepackage{biblatex}
\addbibresource{boatpaper}

\usepackage[usenames,dvipsnames]{xcolor}
\usepackage{booktabs}
\usepackage{graphicx}
\usepackage{etoolbox}
\usepackage{url}
\usepackage[bookmarks]{hyperref}

\setcounter{tocdepth}{3}

\newcommand{\keywords}[1]{\par\addvspace\baselineskip
\noindent\keywordname\enspace\ignorespaces#1}


%%%%%%%%%%%%%%%%%%%%%%%%%%%%%%%%%%%%%%%%%%%%%%%%%%%%%%%%%%%%%%%%%%%%%%%%%%%%%%%%
%% Definitions
%%%%%%%%%%%%%%%%%%%%%%%%%%%%%%%%%%%%%%%%%%%%%%%%%%%%%%%%%%%%%%%%%%%%%%%%%%%%%%%%

\def\pdc{prior-data conflict}

\newcommand{\reals}{\mathbb{R}}
\newcommand{\posreals}{\reals_{>0}}
\newcommand{\posrealszero}{\reals_{\ge 0}}
\newcommand{\naturals}{\mathbb{N}}

\newcommand{\dd}{\,\mathrm{d}}

\newcommand{\mbf}[1]{\mathbf{#1}}
\newcommand{\bs}[1]{\boldsymbol{#1}}
\renewcommand{\vec}[1]{{\bm #1}}

\newcommand{\btheta}{\bs{\theta}}
\newcommand{\bpsi}{\bs{\psi}}
\newcommand{\bbeta}{\bs{\beta}}
\newcommand{\balpha}{\bs{\alpha}}

\newcommand{\uz}{^{(0)}} % upper zero
\newcommand{\un}{^{(n)}} % upper n
\newcommand{\ui}{^{(i)}} % upper i
\newcommand{\uinfty}{^{(\infty)}} % upper infty

\newcommand{\ul}[1]{\underline{#1}}
\newcommand{\ol}[1]{\overline{#1}}

\def\yz{y\uz}
\def\yn{y\un}
\def\yi{y\ui}

\def\yzl{\ul{y}\uz}
\def\yzu{\ol{y}\uz}

\def\ynl{\ul{y}\un}
\def\ynu{\ol{y}\un}

\def\yil{\ul{y}\ui}
\def\yiu{\ol{y}\ui}

\def\nz{n\uz}
\def\nn{n\un}
\def\ni{n\ui}

\def\nzl{\ul{n}\uz}
\def\nzu{\ol{n}\uz}

\def\nnl{\ul{n}\un}
\def\nnu{\ol{n}\un}

\def\nil{\ul{n}\ui}
\def\niu{\ol{n}\ui}

\def\taux{\tau(x)}
\def\ttau{\tilde{\tau}}
\def\ttaux{\ttau(x)}

\def\pz{\psi\uz}
\def\pn{\psi\un}

\def\Y{\mathcal{Y}}
\def\YZ{\Y\uz}
\def\YN{\Y\un}
\def\YI{\Y\ui}
\def\N{\mathcal{N}}
\def\NZ{\N\uz}
\def\NN{\N\un}
\def\PZ{\text{I}\!\Pi\uz}
%\def\PZero{\PZ}
\def\PN{\text{I}\!\Pi\un}
\def\Pinfty{\text{I}\!\Pi\uinfty}
\def\MZ{\mathcal{M}\uz}
\def\MN{\mathcal{M}\un}

\def\Eta{\mathrm{H}}
\def\EZ{\mathrm{H}\uz}
\def\EN{\mathrm{H}\un}

\def\ezl{\ul{\eta}_0}
\def\ezu{\ol{\eta}_0}

\def\ezz{\eta_0\uz}
\def\ezn{\eta_0\un}
\def\eoz{\eta_1\uz}
\def\eon{\eta_1\un}

\def\ezzl{\ul{\eta}_0\uz}
\def\ezzu{\ol{\eta}_0\uz}
\def\eznl{\ul{\eta}_0\un}
\def\ezzn{\ol{\eta}_0\un}

\def\eozl{\ul{\eta}_1\uz}
\def\eozu{\ol{\eta}_1\uz}
\def\eonl{\ul{\eta}_1\un}
\def\eonu{\ol{\eta}_1\un}

\def\eol{\ul{\eta}_1}
\def\eou{\ol{\eta}_1}

\def\czl{\ul{c}\uz}
\def\czu{\ol{c}\uz}

\newcommand{\cdf}{\operatorname{F}}
\newcommand{\p}{\operatorname{P}}
\newcommand{\q}{\operatorname{Q}}
\newcommand{\E}{\operatorname{E}}
\newcommand{\V}{\operatorname{Var}}
\newcommand{\med}{\operatorname{med}} % Median
\newcommand{\modus}{\operatorname{mode}} % Mode
\newcommand{\logit}{\operatorname{logit}} % logit

%\DeclareMathOperator*{\argmin}{arg\,min}
%\DeclareMathOperator*{\argmax}{arg\,max}

\def\El{\ul{\E}}
\def\Eu{\ol{\E}}

\def\Pl{\ul{\p}}
\def\Pu{\ol{\p}}

\newcommand{\ber}{\operatorname{Ber}}   % Bernoulli Distribution
\newcommand{\bin}{\operatorname{Binom}} % Binomial Distribution
\newcommand{\be}{\operatorname{Beta}}   % Beta Distribution
\newcommand{\B}{\operatorname{B}}   % Beta Function



%%%%%%%%%%%%%%%%%%%%%%%%%%%%%%%%%%%%%%%%%%%%%%%%%%%%%%%%%%%%%%%%%%%%%%%%%%%%%%%%
%% Manuscript body
%%%%%%%%%%%%%%%%%%%%%%%%%%%%%%%%%%%%%%%%%%%%%%%%%%%%%%%%%%%%%%%%%%%%%%%%%%%%%%%%


\begin{document}

\mainmatter  % start of an individual contribution

% first the title is needed
\title{Sets of Prior Distributions\\ for Reflecting Prior-Data Conflict\\ and Strong Prior-Data Agreement}

% a short form should be given in case it is too long for the running head
\titlerunning{Sets of Priors for Reflecting Prior-Data Conflict and Strong Prior-Data Agreement}

% the name(s) of the author(s) follow(s) next
%
% NB: Chinese authors should write their first names(s) in front of
% their surnames. This ensures that the names appear correctly in
% the running heads and the author index.
%
\author{Gero Walter\inst{1}%
\thanks{Gero Walter was supported by the Dinalog project
``Coordinated Advanced Maintenance and Logistics Planning for the Process Industries'' (CAMPI).}%
\and Frank P.A.\ Coolen\inst{2}}
%
%\authorrunning{Lecture Notes in Computer Science: Authors' Instructions}
% (feature abused for this document to repeat the title also on left hand pages)

% the affiliations are given next; don't give your e-mail address
% unless you accept that it will be published
\institute{%
School of Industrial Engineering,\\
Eindhoven University of Technology, Eindhoven, NL\\
\url{g.m.walter@tue.nl}
\and
Department of Mathematical Sciences,\\
Durham University, Durham, UK\\
\url{frank.coolen@durham.ac.uk}
}

%
% NB: a more complex sample for affiliations and the mapping to the
% corresponding authors can be found in the file "llncs.dem"
% (search for the string "\mainmatter" where a contribution starts).
% "llncs.dem" accompanies the document class "llncs.cls".
%

\maketitle


\begin{abstract}
Bayesian inference allows to combine observations with (possibly subjective) prior knowledge in the reasoning process.
The choice of a particular prior distribution to represent the available prior knowledge is, however,
often debatable, especially when prior knowledge is limited or data are scarce,
as then posterior inferences are higly dependent on the choice of prior.
Robust Bayesian analysis accounts for this issue by
inquiring whether posterior inferences change substantially
when the prior distribution is varied within a set of distributions that contains all `reasonable' priors.
Similar, but slightly different in scope, is the imprecise probability approach,
formalising the idea that sets of probability distributions
%(or, equivalently, interval-valued previsions or sets of desirable gambles)
should be taken to model prior knowledge more acurrately.
With imprecise probability, modeling prior-data conflict sensitivity is possible:
Models have been proposed where the ranges of posterior inferences
are substantially larger when prior and data are in conflict.
Here we propose a new method for generating parameter prior sets in a conjugate setting
that, in addition to prior-data conflict sensitivity, allow to mirror \emph{strong prior-data agreement},
i.e., the case when prior and data coincide especially well, by increased posterior precision.
Although presented here for the case of binary data only,
it is easily extensible to the general exponential family case.
\keywords{Bayesian inference, strong prior-data agreement, prior-data conflict, imprecise probability, conjugate priors}
\end{abstract}


\section{Introduction}

The Bayesian approach to inference \cite[see, e.g.,][]{2007:robert,2005:ruggeri} 
offers the advantage to combine data and prior knowledge in a unified reasoning process.
The prior knowledge, i.e., information extraneous to data, is usually provided in the form of expert knowledge.

Bayesian inference usually starts with defining a parametric \emph{sample model},
denoted by $f(\vec{x} \mid \vartheta)$,
a conditional distribution of data $\vec{x} = (x_1, \ldots, x_n)$ given the value for a parameter $\vartheta$.
Prior information is then encoded by a so-called \emph{prior distribution} $p(\vartheta)$,
expressing the expert's opinion on probable values of $\vartheta$.
In light of the sample $\vec{x}$, the prior distribution is updated by Bayes' Rule
to obtain the so-called \emph{posterior distribution}
\begin{align}
\label{eq:bayesrule}
p(\vartheta\mid\vec{x}) \propto f(\vec{x}\mid\vartheta) \cdot p(\vartheta)\,.
\end{align}
The posterior distribution is understood to comprise all information from the sample and the prior knowledge.
It therefore underlies all further inferences on the parameter $\vartheta$,
like point estimators, interval estimators,
or the \emph{posterior predictive distribution},
giving the distribution of further observations based on the posterior.

However, the choice for the prior distribution to encode the given expert knowledge is often debatable,
and a specific choice of prior is difficult to justify.
A way to deal with this is to employ sensitivity analysis,
i.e., studying the effect of different choices of prior distribution on the quantities of interest.
This idea has been explored in systematic sensitivity analysis, or robust Bayesian methods;
for an overview on this approach, see, e.g.,
\cite{1994:berger} or \cite{2000:rios}. %\citeNP{2005:ruggeri}, \citeNP{2000:bergerinsuaruggeri}

The work we present here can be seen as belonging to the robust Bayesian approach, as our work uses sets of priors.
However, our focus and interpretation is slightly different,
as we consider the result of our procedure, sets of posterior distributions, as the proper result,
while a robust Bayesian would base his analyses on a single posterior from the set
in case (s)he was able to conclude that quantities of interest are not `too sensitive' to the choice of prior.
In contrast, our viewpoint is rooted in the theory of imprecise or interval probability \cite{itip,1991:walley},
where sets of distributions are used to express the precision of probability statements themselves:
the smaller the set of posteriors, the more precise the probability statement.


% from abstract esrelpaper
A problem that can arise in Bayesian inference is called prior-data conflict:
from the viewpoint of the prior, the observed data seem very surprising,
i.e., the information from data is in conflict with the prior assumptions.
It has been recognised that models based on conjugate priors can be insensitive to prior-data conflict,
in the sense that the spread of the posterior distribution does not increase in case of such a conflict,
thus conveying a false sense of certainty by communicating that we know what's going on quite precisely when in fact we do not.


motivation: clever choice of prior sets

***emphazise that this is basically an idea for defining sets of priors***

First, we will briefly characterise the novel parametrisation of canoncial conjugate priors this approach relies on.
To keep things simple, we restrict ourselves here for the case of the Beta-Binomial model (see Section~\ref{sec:beta-binom}),
but the approach is generalisable to arbitrary canonical conjugate priors.
Then we will suggest a shape in this parametrisation that accomplishes
both \pdc\ sensitivity and `bonus precision' in case of strong prior-data agreement.
We propose a parametric description for such a shape
and show that it indeed leads to the desired properties.



The combination of a Binomial observation model with a Beta prior is often called Beta-Binomial model.


% ------------ bibliography -------------

\printbibliography


\end{document}
