\documentclass[runningheads,a4paper]{llncs}

\usepackage[utf8]{inputenc}
\usepackage[OT1]{fontenc}
\usepackage[english]{babel}

\usepackage{amsfonts}
\usepackage{amssymb}
%\usepackage{amsthm}
\usepackage{amssymb}

\usepackage[usenames,dvipsnames]{xcolor}
\usepackage{booktabs}
\usepackage{graphicx}
\usepackage{etoolbox}
\usepackage{url}
\usepackage[bookmarks]{hyperref}

\setcounter{tocdepth}{3}

\newcommand{\keywords}[1]{\par\addvspace\baselineskip
\noindent\keywordname\enspace\ignorespaces#1}


%%%%%%%%%%%%%%%%%%%%%%%%%%%%%%%%%%%%%%%%%%%%%%%%%%%%%%%%%%%%%%%%%%%%%%%%%%%%%%%%
%% Definitions
%%%%%%%%%%%%%%%%%%%%%%%%%%%%%%%%%%%%%%%%%%%%%%%%%%%%%%%%%%%%%%%%%%%%%%%%%%%%%%%%

\def\pdc{prior-data conflict}

\newcommand{\reals}{\mathbb{R}}
\newcommand{\posreals}{\reals_{>0}}
\newcommand{\posrealszero}{\reals_{\ge 0}}
\newcommand{\naturals}{\mathbb{N}}

\newcommand{\dd}{\,\mathrm{d}}

\newcommand{\mbf}[1]{\mathbf{#1}}
\newcommand{\bs}[1]{\boldsymbol{#1}}

\newcommand{\btheta}{\bs{\theta}}
\newcommand{\bpsi}{\bs{\psi}}
\newcommand{\bbeta}{\bs{\beta}}
\newcommand{\balpha}{\bs{\alpha}}

\newcommand{\uz}{^{(0)}} % upper zero
\newcommand{\un}{^{(n)}} % upper n
\newcommand{\ui}{^{(i)}} % upper i
\newcommand{\uinfty}{^{(\infty)}} % upper infty

\newcommand{\ul}[1]{\underline{#1}}
\newcommand{\ol}[1]{\overline{#1}}

\def\yz{y\uz}
\def\yn{y\un}
\def\yi{y\ui}

\def\yzl{\ul{y}\uz}
\def\yzu{\ol{y}\uz}

\def\ynl{\ul{y}\un}
\def\ynu{\ol{y}\un}

\def\yil{\ul{y}\ui}
\def\yiu{\ol{y}\ui}

\def\nz{n\uz}
\def\nn{n\un}
\def\ni{n\ui}

\def\nzl{\ul{n}\uz}
\def\nzu{\ol{n}\uz}

\def\nnl{\ul{n}\un}
\def\nnu{\ol{n}\un}

\def\nil{\ul{n}\ui}
\def\niu{\ol{n}\ui}

\def\taux{\tau(x)}
\def\ttau{\tilde{\tau}}
\def\ttaux{\ttau(x)}

\def\pz{\psi\uz}
\def\pn{\psi\un}

\def\Y{\mathcal{Y}}
\def\YZ{\Y\uz}
\def\YN{\Y\un}
\def\YI{\Y\ui}
\def\N{\mathcal{N}}
\def\NZ{\N\uz}
\def\NN{\N\un}
\def\PZ{\text{I}\!\Pi\uz}
%\def\PZero{\PZ}
\def\PN{\text{I}\!\Pi\un}
\def\Pinfty{\text{I}\!\Pi\uinfty}
\def\MZ{\mathcal{M}\uz}
\def\MN{\mathcal{M}\un}

\def\Eta{\mathrm{H}}
\def\EZ{\mathrm{H}\uz}
\def\EN{\mathrm{H}\un}

\def\ezl{\ul{\eta}_0}
\def\ezu{\ol{\eta}_0}

\def\ezz{\eta_0\uz}
\def\ezn{\eta_0\un}
\def\eoz{\eta_1\uz}
\def\eon{\eta_1\un}

\def\ezzl{\ul{\eta}_0\uz}
\def\ezzu{\ol{\eta}_0\uz}
\def\eznl{\ul{\eta}_0\un}
\def\ezzn{\ol{\eta}_0\un}

\def\eozl{\ul{\eta}_1\uz}
\def\eozu{\ol{\eta}_1\uz}
\def\eonl{\ul{\eta}_1\un}
\def\eonu{\ol{\eta}_1\un}

\def\eol{\ul{\eta}_1}
\def\eou{\ol{\eta}_1}

\def\czl{\ul{c}\uz}
\def\czu{\ol{c}\uz}

\newcommand{\cdf}{\operatorname{F}}
\newcommand{\p}{\operatorname{P}}
\newcommand{\q}{\operatorname{Q}}
\newcommand{\E}{\operatorname{E}}
\newcommand{\V}{\operatorname{Var}}
\newcommand{\med}{\operatorname{med}} % Median
\newcommand{\modus}{\operatorname{mode}} % Mode
\newcommand{\logit}{\operatorname{logit}} % logit

%\DeclareMathOperator*{\argmin}{arg\,min}
%\DeclareMathOperator*{\argmax}{arg\,max}

\def\El{\ul{\E}}
\def\Eu{\ol{\E}}

\def\Pl{\ul{\p}}
\def\Pu{\ol{\p}}

\newcommand{\ber}{\operatorname{Ber}}   % Bernoulli Distribution
\newcommand{\bin}{\operatorname{Binom}} % Binomial Distribution
\newcommand{\be}{\operatorname{Beta}}   % Beta Distribution
\newcommand{\B}{\operatorname{B}}   % Beta Function



%%%%%%%%%%%%%%%%%%%%%%%%%%%%%%%%%%%%%%%%%%%%%%%%%%%%%%%%%%%%%%%%%%%%%%%%%%%%%%%%
%% Manuscript body
%%%%%%%%%%%%%%%%%%%%%%%%%%%%%%%%%%%%%%%%%%%%%%%%%%%%%%%%%%%%%%%%%%%%%%%%%%%%%%%%


\begin{document}

\mainmatter  % start of an individual contribution

% first the title is needed
\title{Sets of Prior Distributions\\ for Reflecting Prior-Data Conflict\\ and Strong Prior-Data Agreement}

% a short form should be given in case it is too long for the running head
\titlerunning{Sets of Priors for Reflecting Prior-Data Conflict and Strong Prior-Data Agreement}

% the name(s) of the author(s) follow(s) next
%
% NB: Chinese authors should write their first names(s) in front of
% their surnames. This ensures that the names appear correctly in
% the running heads and the author index.
%
\author{Gero Walter\inst{1}%
\thanks{Gero Walter was supported by the Dinalog project
``Coordinated Advanced Maintenance and Logistics Planning for the Process Industries'' (CAMPI).}%
\and Frank P.A.\ Coolen\inst{2}}
%
%\authorrunning{Lecture Notes in Computer Science: Authors' Instructions}
% (feature abused for this document to repeat the title also on left hand pages)

% the affiliations are given next; don't give your e-mail address
% unless you accept that it will be published
\institute{%
School of Industrial Engineering,\\
Eindhoven University of Technology, Eindhoven, NL\\
\url{g.m.walter@tue.nl}
\and
Department of Mathematical Sciences,\\
Durham University, Durham, UK\\
\url{frank.coolen@durham.ac.uk}
}

%
% NB: a more complex sample for affiliations and the mapping to the
% corresponding authors can be found in the file "llncs.dem"
% (search for the string "\mainmatter" where a contribution starts).
% "llncs.dem" accompanies the document class "llncs.cls".
%

\maketitle


\begin{abstract}
The Bayesian approach to inference offers the advantage
to include prior knowledge along with the data in the reasoning process.
The choice of a particular prior distribution to represent the available prior knowledge is, however,
often debatable, especially when prior knowledge is limited or data are scarce,
as then posterior inferences are higly dependent on the choice of prior.
Robust Bayesian analysis accounts for this issue by
inquiring whether posterior inferences change substantially
when the prior distribution is varied within a set of distributions that contains all `reasonable' priors.
Similar, but slightly different in scope, is the imprecise probability approach,
formalising the idea that sets of probability distributions
%(or, equivalently, interval-valued previsions or sets of desirable gambles)
should be taken to model prior knowledge more acurrately.
With imprecise probability, modeling prior-data conflict sensitivity is possible:
Models have been proposed where the ranges of posterior inferences
are substantially larger when prior and data are in conflict.
Here we propose a new method for generating parameter prior sets in a conjugate setting
that, in addition to prior-data conflict sensitivity, allow to mirror \emph{strong prior-data agreement},
i.e., the case when prior and data coincide especially well, by increased posterior precision.
Although presented here for the case of binary data only,
it is easily extensible to the general exponential family case.
\keywords{Bayesian inference, strong prior-data agreement, prior-data conflict, imprecise probability, conjugate priors}
\end{abstract}


\section{Introduction}



\end{document}
